\documentclass{article}
            \usepackage{amsthm}

            \newtheorem{definition}{Definition}
            \newtheorem{theorem}{Theorem}
            \newtheorem{axiom}{Axiom}


            \begin{document}

            \title{Reductio5}
            \maketitle
            

\begin{axiom}
\begin{equation}
{{{{\left( {{x} = {y}} \; \mathrm{ and } \; {{y} = {z}}\right)} \Longrightarrow {{x} = {z}}}}}
\end{equation}
\end{axiom}

\begin{axiom}
\begin{equation}
{1} = {a}
\end{equation}
\end{axiom}

\begin{axiom}
\begin{equation}
\neg {\left( {1} = {2}\right)}
\end{equation}
\end{axiom}

\begin{theorem}
\begin{equation}
\neg {{a} = {2}}
\end{equation}

\begin{proof}
We will assume the contrary:

\begin{equation}
{a} = {2}
\end{equation}

Next, we use the transitivity of equality.

\begin{equation}
\forall {y} : {\left( \forall {z} : {\left( {{{1} = {y}} \; \mathrm{ and } \; {{y} = {z}}} \Longrightarrow {{1} = {z}}\right)}\right)}
\end{equation}

\begin{equation}
\forall {z} : {\left( {{{1} = {a}} \; \mathrm{ and } \; {{a} = {z}}} \Longrightarrow {{1} = {z}}\right)}
\end{equation}

\begin{equation}
{{{1} = {a}} \; \mathrm{ and } \; {{a} = {2}}} \Longrightarrow {{1} = {2}}
\end{equation}

Using the axiom  a = 1 as well as our initial assumtion, this reduces to:

\begin{equation}
{1} = {2}
\end{equation}

Obvoiously, a statement and its negation cannot both hold true.

\begin{equation}
{{{1} = {2}} \; \mathrm{ and } \; {\neg {{1} = {2}}}} \Longrightarrow {\mathrm{False}}
\end{equation}

However, we just deduced the former equality, and we presupposed the latter. This leads to a contradiction.

\begin{equation}
\mathrm{False}
\end{equation}

This means our assumption must have been wrong.

\begin{equation}
\neg {{a} = {2}}
\end{equation}

\end{proof}
\end{theorem}

\end{document}